%########################################
%CS 461 CAPSTONE 
%GROUP 20 - Griffin Gonsalves, Paul Kwak, Shawn Cross
%FALL 2016
%########################################
\documentclass[letterpaper, 10pt, draftclsnofoot, compsoc, onecolumn]{IEEEtran}
\usepackage[latin1]{inputenc}
\usepackage[T1]{fontenc}
\usepackage[english]{babel}
\usepackage{amsmath}
\usepackage{amssymb,amsfonts,textcomp}
\usepackage{color}
\usepackage{array}
\usepackage{supertabular}
\usepackage{hhline}
%\usepackage{biblatex}
\usepackage{cite}


\usepackage{hyperref}
%\usepackage{digsig} %not a default package... just for signature field on final page.
\usepackage{comment}

\hypersetup{pdftex, colorlinks=true, linkcolor=black, citecolor=blue, filecolor=blue, urlcolor=blue, pdftitle=SYSTEMS AND SOFTWARE REQUIREMENTS SPECIFICATION (SSRS) TEMPLATE, pdfauthor=Clinton Jeffery, pdfsubject=, pdfkeywords=}
\usepackage[pdftex]{graphicx}
% Outline numbering
\setcounter{secnumdepth}{4}
\renewcommand\thesection{\arabic{section}}
\renewcommand\thesubsection{\arabic{section}.\arabic{subsection}}
\renewcommand\thesubsubsection{\arabic{section}.\arabic{subsection}.\arabic{subsubsection}}
\renewcommand\theparagraph{\arabic{section}.\arabic{subsection}.\arabic{subsubsection}.\arabic{paragraph}}

\parindent0pt
\parskip 1.5ex plus 0.2ex minus 0.1ex
\makeatletter

\def\subsubsection{\@startsection{subsubsection}% name
                                 {3}% level
                                 {\z@}% indent (formerly \parindent)
                                 {1ex plus 0.1ex minus 0.1ex}% before skip
                                 {1ex}% after skip
                                 {\normalfont\normalsize}}% style

\newcommand\arraybslash{\let\\\@arraycr}
\makeatother
% Page layout (geometry)
\setlength\voffset{-1in}
\setlength\hoffset{-1in}
\setlength\topmargin{0.5in}
\setlength\oddsidemargin{.75in}
\setlength\evensidemargin{.75in}
\setlength\textheight{8.278in}
\setlength\textwidth{6.5in}
\setlength\footskip{0.561in}
\setlength\headheight{0.5in}
\setlength\headsep{0.461in}
% Footnote rule
\setlength{\skip\footins}{0.0469in}
\renewcommand\footnoterule{\vspace*{-0.0071in}\setlength\leftskip{0pt}\setlength\rightskip{0pt plus 1fil}\noindent\textcolor{black}{\rule{0.25\columnwidth}{0.0071in}}\vspace*{0.0398in}}
% Pages styles
\makeatletter
\newcommand\ps@Standard{
  \renewcommand\@oddhead{\hfill }
  \renewcommand\@evenhead{\@oddhead}
  \renewcommand\@oddfoot{\foreignlanguage{english}{\textcolor{black}{SSRS Page }}\foreignlanguage{english}{\textcolor{black}{\thepage{}}}}
  \renewcommand\@evenfoot{\@oddfoot}
  \renewcommand\thepage{\arabic{page}}
}
\newcommand\ps@FirstPage{
  \renewcommand\@oddhead{}
  \renewcommand\@evenhead{\@oddhead}
  \renewcommand\@oddfoot{}
  \renewcommand\@evenfoot{\@oddfoot}
  \renewcommand\thepage{\arabic{page}}
}
\makeatother
\pagestyle{Standard}
%\setlength\tabcolsep{1mm}
\renewcommand\arraystretch{1.3}
% footnotes configuration
\makeatletter
\renewcommand\thefootnote{\arabic{footnote}}
\makeatother
\title{Forge VR Explorer Requirements}
\author{Shawn Cross, Griffin Gonsalves, Paul Kwak}
\date{2016-11-3}


\begin{document}

\clearpage\setcounter{page}{1}\pagestyle{Standard}
\thispagestyle{FirstPage}

\bigskip

{\centering\selectlanguage{english}\bfseries\color{black}
Design Document
\par}


\bigskip

{\centering\selectlanguage{english}\bfseries\color{black}
November 17 2016
\par}
\bigskip
\bigskip
\bigskip
\bigskip
\bigskip
\bigskip
\bigskip
\bigskip
\bigskip
\bigskip
\bigskip
\bigskip
%\begin{center}
%	\includegraphics[scale=0.8]{forge_logo.png}
%\end{center}


\vfill
{\centering\selectlanguage{english}\bfseries\color{black}
Abstract
\par}

{\centering\selectlanguage{english}\mdseries\color{black}
	The Forge VR Explorer branches from an Autodesk prototype project called Vrok-It, which is a simple web-based 3D 
	model viewer and mobile virtual reality (VR) explorer. The project will expand upon its ability to display uploaded 3D 
	models in browser and in VR, and improve its accessibility. Conventionally, viewing 3D models in VR is a challenge if 
	you have model files on many devices, or have a headset that only works in conjunction with a smart-phone. The 
	Forge VR Explorer aims to do this by utilizing a web-based software that uses the features of the Autodesk Forge API. 
	The project will also be expanded with new ideas and stretch goals as the project is developed.
\par}
\clearpage
{\centering\selectlanguage{english}\bfseries\color{black}
TABLE OF CONTENTS
\par}

\bigskip

\setcounter{tocdepth}{2}
\renewcommand\contentsname{}
\tableofcontents

\bigskip
\clearpage


\section{Overview}

\subsection{Scope}

\subsection{Purpose}

\subsection{Intended audience}

\subsection{conformance}

\section{Definitions}
	%\begin{description}
	%\end{description}
	
\section{Conceptual model for software design descriptions}

\subsection{Software design in context}

\subsection{Software design description within the life cycle}

\subsubsection{Influences on SDD preparation}

\subsubsection{influences on software life cycle products}

\subsubsection{Design verification and design role validation}

\section{Design description information content}

\subsection{Introduction}

\subsection{SDD identification}

\subsection{Design stakeholders and their concerns}

\subsection{Design views}

\subsection{Design viewpoints}

\subsection{Design elements}

\subsubsection{Design entities}

\subsubsection{Design attributes}

\paragraph{Name attribute}

\paragraph{Type attribute}

\paragraph{Purpose attribute}

\paragraph{Author attribute}

\subsubsection{Design relationships}

\subsubsection{Design constraints}

\subsection{Design overlays}

\subsection{Design rationale}

\subsection{Design languages}

\section{Design viewpoints}

\subsection{Introduction}

\subsection{Context viewpoint}

\subsubsection{Design concerns}

\subsubsection{Design elements}

\subsubsection{Example languages}

\subsection{composition viewpoint}

\subsubsection{Design concerns}

\subsubsection{Design elements}

\paragraph{Function attributes}

\paragraph{Subordinates attributes}

\subsubsection{Example languages}

\subsection{Logical viewpoints}

\subsubsection{Design concerns}

\subsubsection{Design elements}

\subsubsection{Example languages}

\subsection{Dependency viewpoint}

\subsubsection{Design concerns}

\subsubsection{Design elements}

\paragraph{Dependencies attribute}

\subsubsection{Example languages}

\subsection{Information viewpoint}

\subsubsection{Design concerns}

\subsubsection{Design elements}

\paragraph{Data attribute}

\subsubsection{Example languages}

\subsection{Patterns use viewpoint}

\subsubsection{Design concerns}

\subsubsection{Design elements}

\subsubsection{Example languages}

\subsection{Interface viewpoint}

\subsubsection{Design concerns}

\subsubsection{Design elements}

\paragraph{Interface attribute}

\subsubsection{Example languages}

\subsection{Structure viewpoint}

\subsubsection{Design concerns}

\subsubsection{Design elements}

\subsubsection{Example languages}

\subsection{Interaction viewpoint}

\subsubsection{Design concerns}

\subsubsection{Design elements}

\subsubsection{Examples}

\subsection{State dynamics viewpoint}

\subsubsection{Design concerns}

\subsubsection{Design elements}

\subsubsection{Example languages}

\subsection{Algorithm viewpoint}

\subsubsection{Design concerns}

\subsubsection{Design elements}

\subsubsection{Processing attribute}

\subsubsection{Examples}

\subsection{Resource viewpoint}

\subsubsection{Design concerns}

\subsubsection{Design elements}

\paragraph{Resources attributes}

\subsubsection{Examples}

%\bibliography{bibfile}
%\bibliographystyle{IEEETran}
\end{document}


%@misc{a javascript library for building user interfaces - react,
 %title={A JavaScript library for building user interfaces - React},
 %url={https://facebook.github.io/react/},
 %journal={A JavaScript library for building user interfaces - React}, 
 %publisher={Facebook.com}}

