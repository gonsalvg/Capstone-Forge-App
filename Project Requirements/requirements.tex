%########################################
%CS 461 CAPSTONE 
%GROUP 20 - Griffin Gonsalves, Paul Kwak, Shawn Cross
%FALL 2016
%########################################
\documentclass[letterpaper, 10pt, draftclsnofoot, compsoc, onecolumn]{IEEEtran}
\usepackage[latin1]{inputenc}
\usepackage[T1]{fontenc}
\usepackage[english]{babel}
\usepackage{amsmath}
\usepackage{amssymb,amsfonts,textcomp}
\usepackage{color}
\usepackage{array}
\usepackage{supertabular}
\usepackage{hhline}

\usepackage{hyperref}
%\usepackage{digsig} %not a default package... just for signature field on final page.
\usepackage{comment}

\hypersetup{pdftex, colorlinks=true, linkcolor=blue, citecolor=blue, filecolor=blue, urlcolor=blue, pdftitle=SYSTEMS AND SOFTWARE REQUIREMENTS SPECIFICATION (SSRS) TEMPLATE, pdfauthor=Clinton Jeffery, pdfsubject=, pdfkeywords=}
\usepackage[pdftex]{graphicx}
% Outline numbering
\setcounter{secnumdepth}{3}
\renewcommand\thesection{\arabic{section}}
\renewcommand\thesubsection{\arabic{section}.\arabic{subsection}}
\renewcommand\thesubsubsection{\arabic{section}.\arabic{subsection}.\arabic{subsubsection}}

\parindent0pt
\parskip 1.5ex plus 0.2ex minus 0.1ex
\makeatletter

\def\subsubsection{\@startsection{subsubsection}% name
                                 {3}% level
                                 {\z@}% indent (formerly \parindent)
                                 {1ex plus 0.1ex minus 0.1ex}% before skip
                                 {1ex}% after skip
                                 {\normalfont\normalsize}}% style

\newcommand\arraybslash{\let\\\@arraycr}
\makeatother
% Page layout (geometry)
\setlength\voffset{-1in}
\setlength\hoffset{-1in}
\setlength\topmargin{0.5in}
\setlength\oddsidemargin{.75in}
\setlength\evensidemargin{.75in}
\setlength\textheight{8.278in}
\setlength\textwidth{6.5in}
\setlength\footskip{0.561in}
\setlength\headheight{0.5in}
\setlength\headsep{0.461in}
% Footnote rule
\setlength{\skip\footins}{0.0469in}
\renewcommand\footnoterule{\vspace*{-0.0071in}\setlength\leftskip{0pt}\setlength\rightskip{0pt plus 1fil}\noindent\textcolor{black}{\rule{0.25\columnwidth}{0.0071in}}\vspace*{0.0398in}}
% Pages styles
\makeatletter
\newcommand\ps@Standard{
  \renewcommand\@oddhead{\hfill }
  \renewcommand\@evenhead{\@oddhead}
  \renewcommand\@oddfoot{\foreignlanguage{english}{\textcolor{black}{SSRS Page }}\foreignlanguage{english}{\textcolor{black}{\thepage{}}}}
  \renewcommand\@evenfoot{\@oddfoot}
  \renewcommand\thepage{\arabic{page}}
}
\newcommand\ps@FirstPage{
  \renewcommand\@oddhead{}
  \renewcommand\@evenhead{\@oddhead}
  \renewcommand\@oddfoot{}
  \renewcommand\@evenfoot{\@oddfoot}
  \renewcommand\thepage{\arabic{page}}
}
\makeatother
\pagestyle{Standard}
%\setlength\tabcolsep{1mm}
\renewcommand\arraystretch{1.3}
% footnotes configuration
\makeatletter
\renewcommand\thefootnote{\arabic{footnote}}
\makeatother
\title{Forge VR Explorer Requirements}
\author{Shawn Cross, Griffin Gonsalves, Paul Kwak}
\date{2016-11-3}



\begin{document}

\clearpage\setcounter{page}{1}\pagestyle{Standard}
\thispagestyle{FirstPage}

\bigskip

{\centering\selectlanguage{english}\bfseries\color{black}
Forge VR Explorer Requirements
\par}


\bigskip

{\centering\selectlanguage{english}\bfseries\color{black}
November 4 2016
\par}
\bigskip
\bigskip
\bigskip
\bigskip
\bigskip
\bigskip
\bigskip
\bigskip
\bigskip
\bigskip
\bigskip
\bigskip
\begin{center}
	\includegraphics[scale=0.8]{forge_logo.png}
\end{center}


\vfill
{\centering\selectlanguage{english}\bfseries\color{black}
Abstract
\par}

{\centering\selectlanguage{english}\mdseries\color{black}
	The Forge VR Explorer branches from an Autodesk prototype project called Vrok-It, which is a simple web-based 3D 
	model viewer and mobile virtual reality (VR) explorer. The project will expand upon its ability to display uploaded 3D 
	models in browser and in VR, and improve its accessibility. Conventionally, viewing 3D models in VR is a challenge if 
	you have model files on many devices, or have a headset that only works in conjunction with a smartphone. The 
	Forge VR Explorer aims to do this by utilizing a web-based software that uses the features of the Autodesk Forge API. 
	The project will also be expanded with new ideas and stretch goals as the project is developed.
\par}

\bigskip
\bigskip

\clearpage{\centering\selectlanguage{english}\bfseries\color{black}
SYSTEMS AND SOFTWARE \ REQUIREMENTS SPECIFICATION (SSRS) FOR 
\par}

\bigskip

{\centering\selectlanguage{english}\bfseries\color{black}
Forge VR Explorer
\par}

\bigskip
\bigskip
\bigskip

\begin{figure}
\centering
%\includegraphics[width=3.4362in,height=0.6134in]{SSRSTemplateA2-img1.png}
\end{figure}

\bigskip
\bigskip

{\centering\selectlanguage{english}\bfseries\color{black}
Version 1.0
\par}

{\centering\selectlanguage{english}\bfseries\color{black}
11/4/2016
\par}


\bigskip
\bigskip

{\centering\selectlanguage{english}\bfseries\color{black}
Prepared for:
\par}

{\centering\selectlanguage{english}\bfseries\color{black}
Patti Vrobel, Autodesk
\par}


\bigskip
\bigskip

{\centering\selectlanguage{english}\bfseries\color{black}
Prepared by:  
\par}

{\centering\selectlanguage{english}\bfseries\color{black}
Shawn Cross, Griffin Gonsalves, Paul Kwak
\par}

{\centering\selectlanguage{english}\bfseries\color{black}
Oregon State University
\par}

{\centering\selectlanguage{english}\bfseries\color{black}
Corvallis, OR \ 97331
\par}

\clearpage{\centering\selectlanguage{english}\bfseries\color{black}
\foreignlanguage{english}{\MakeUppercase{\ }}\foreignlanguage{english}{\MakeUppercase{Forge VR Explorer SSRS}}
\par}

{\centering\selectlanguage{english}\bfseries\color{black}
TABLE OF CONTENTS
\par}

\bigskip

{\selectlanguage{english}\bfseries\color{black}
Section\ \ Page}

\setcounter{tocdepth}{9}
\renewcommand\contentsname{}
\tableofcontents

\bigskip
\clearpage

\section[Introduction]{\selectlanguage{english}\rmfamily\bfseries\color{black}
Introduction}
	The system being developed is intended to be a place in which users with CAD files can go and easily view those files 
	in both a 3D viewer and optionally in VR. This provides a solution to those who don't have access to expensive CAD 
	programs or fancy VR headsets as it will be a free to use web services that is usable with entry level VR equipment. 
	The intended user for this project is a person that would likely not experienced with CAD software and would not have 
	access to software that supports CAD files.  

\subsection[SCOPE]{\selectlanguage{english}\rmfamily\bfseries\color{black}
SCOPE}
	The scope of this project includes developing new software solutions over the course of roughly 3 months. This software
 	should allow for the upload of a 3D model file to be viewed in browser and then also into a VR environment. Having a 
	platform that allows for simple 3D viewing on screen into a transition to VR would allow many to have access to such 
	functionality. 

\subsection[DEFINITIONS, ACRONYMS, AND
ABBREVIATIONS]{\selectlanguage{english}\rmfamily\bfseries\color{black}
DEFINITIONS, ACRONYMS, AND ABBREVIATIONS}
	\begin{flushleft}
	\textbf{CAD}: Computer Aided design is software used to design and view 3D models.

	\textbf{CAD file}: The type of files that can be uploaded to the website for viewing in the Forge viewer.
	We will  narrow down what types of files can be used as we progress through the development process.

	\textbf{FORGE}: A collection of CAD API services provided by Autodesk.

	\textbf{Forge Viewer}: This is one of the Forge APIs. It displays 3D models from CAD files and also allows
	for user interaction.

	\textbf{VR}: Acronym for virtual reality, typically a peripheral device or smartphone
	\end{flushleft} 
	
\subsection[REFERENCES]{\selectlanguage{english}\rmfamily\bfseries\color{black}
REFERENCES}
%Sources}
% maybe use a bibtex file for this when closer to completion
	\begin{flushleft}
	\url{https://developer.autodesk.com/}\newline

	\url{https://github.com/KeanW/vr-party}\newline

	\url{Vrok.it}\newline
	\end{flushleft}

\clearpage

\section[OVERALL
DESCRIPTION]{\selectlanguage{english}\rmfamily\bfseries\color{black}
OVERALL DESCRIPTION}

\subsection[PRODUCT
PERSPECTIVE]{\selectlanguage{english}\rmfamily\bfseries\color{black}
PRODUCT PERSPECTIVE}
	The product would be jump-started from the Vrok-It platform that has already been created. 
	We would adding features to the product that will enhance the overall usability of the current system. 

\subsection[PRODUCT
FUNCTIONS]{\selectlanguage{english}\rmfamily\bfseries\color{black}
PRODUCT FUNCTIONS}
	\begin{flushleft}
	1. The user is able to pick from a list of 3D models or upload one of their own CAD files. \newline

	2. Once a model is chosen or a file is uploaded the 3D models will be seen in the Forge viewer at the 
	center of the website.\newline

	3. "Viewable" 3D models can be interacted with by the user in the Forge viewer in multiple ways.\newline

	4. The user will have the ability to connect a smartphone to the website through the use of a QR scanner
	and web servers. \newline

	5. Once connected to a smartphone, the current model should be able to be viewed in VR with a the use
	of a VR headset such as Google Cardboard. 
	\end{flushleft}

\subsection[USER
CHARACTERISTICS]{\selectlanguage{english}\rmfamily\bfseries\color{black}
USER CHARACTERISTICS}
	When finished this product should be usable by anyone that has access to a CAD file. If the user is want to use the VR 
	portion of the website then they will need access to some VR headset. If the user does have access to a VR headset they 
	should not need any extra Knowledge other than how to use the headset.  

\subsection[SYSTEM LEVEL (NON{}-FUNCTIONAL)
REQUIREMENTS]{\selectlanguage{english}\rmfamily\bfseries\color{black}
SYSTEM LEVEL (NON-FUNCTIONAL) REQUIREMENTS}

\subsubsection[Software
Interfaces]{\selectlanguage{english}\rmfamily\bfseries\color{black}
Software Interfaces}
	Interface between the computer and the website, for the uploading of CAD models from a user's hard drive into the 
	website for viewing. Input will be CAD models while the output will be the website window in which the users model 
	is displayed. Interface between the website and the device the user wishes to view the model on. Currently vrok.it uses a 
	QR code to accomplish this-Input is the QR code scanned by the phone, output is the manipulatable model in a environment 
	for viewing. The User interface for viewing model on smartphone is another interface. A user's touchscreen that gathers 
	input and the output is the manipulation they make to the model.

\subsubsection[User
Interfaces]{\selectlanguage{english}\rmfamily\bfseries\color{black}
User Interfaces}
	The main user interface will be the website(TBD) in which the user will be able to upload a CAD model and then view it in 
	the Forge viewer at the center of the webpage. The user will be able interact with the model through use of their 
	mouse. They will have options to rotate the model look at an exploded view of the model, along with other options. 

\section[SPECIFIC
REQUIREMENTS]{\selectlanguage{english}\rmfamily\bfseries\color{black}
SPECIFIC REQUIREMENTS}
\bigskip

\subsection[SYSTEM
FEATURES]{\selectlanguage{english}\rmfamily\bfseries\color{black}
SYSTEM FEATURES}
\medskip

\subsubsection[{File Uploading}]{\rmfamily\bfseries\color{black}
	The ability for the user to upload a CAD file of their choosing
}


\paragraph[Introduction/Purpose of this feature]
{\selectlanguage{english}\rmfamily\bfseries\color{black} Introduction/Purpose }
	This will allow the user to be able to upload any CAD files in which they have access to the website.

\paragraph[Input/Output sequence:]{\selectlanguage{english}\rmfamily\bfseries\color{black}
Input/Output sequence }
	The user will select a file from their computer that they would like to upload to the website. After finishing the upload process 
	the model that was in the file should be viewable on the website. 

\paragraph[Design constraints of this
feature]{\selectlanguage{english}\rmfamily\bfseries\color{black} Design
constraints }
	The file must be a CAD specific file and the user must have access to it on their machine. 

\paragraph[Performance requirements of this
feature]{\selectlanguage{english}\rmfamily\bfseries\color{black}
Performance requirements }
	The file must be able to be uploaded in a reasonable amount of time depending on the size of the file. This should take at 
	most 1 minute. 


\paragraph[Detailed functional requirements of this
feature]{\selectlanguage{english}\rmfamily\bfseries\color{black}
Detailed functional requirements }
	The user should be able to upload his/her file to the website using a standard popup window. The software must determine if 
	the file is a reasonable size(<50MB), and is an accepted format. After the file is uploaded, it is then made available to the 
	other components of the software to be used.

%end of upload feature

\subsubsection[{Viewable Model}]{\selectlanguage{english}\rmfamily\bfseries\color{black}
	The user should be able to see and interact with their model
}

\paragraph[Introduction/Purpose of this
feature]{\selectlanguage{english}\rmfamily\bfseries\color{black}
Introduction/Purpose }
	The user should be able to see the model that they have chosen or uploaded to the site in the Forge viewer. They 
	should also be able to interact with that model in the Forge viewer. 

\paragraph[Input/Output sequence for this
feature]{\selectlanguage{english}\rmfamily\bfseries\color{black}
Input/Output sequence }
	The user will chose from the list of predefined models or upload their own model. The model will now be view able in the large 
	model viewer and the user should be able to interact with that model.

\paragraph[Design constraints of this
feature]{\selectlanguage{english}\rmfamily\bfseries\color{black} Design constraints }
	The model must not be so large or detailed that the website can not render it.


\paragraph[Performance requirements of this
feature]{\selectlanguage{english}\rmfamily\bfseries\color{black}
Performance requirements }
	The model should be displayed in full in the Forge viewer and should not be hard to interact with. 

\paragraph[Detailed functional requirements of this
feature]{\selectlanguage{english}\rmfamily\bfseries\color{black}
Detailed functional requirements }
	The Forge viewer will present the user's model in a window set into the webpage. This window must be interactive, and display the 
	user's model correctly. The viewer should be able to display a reasonably sized(<50MB) model. 

%end of feature 2

\subsubsection[{Smartphone Connection}]{\selectlanguage{english}\rmfamily\bfseries\color{black} User should be able to connect their 
	device to the project website through the use of a QR scanner
}


\paragraph[Introduction/Purpose of this
feature]{\selectlanguage{english}\rmfamily\bfseries\color{black}
Introduction/Purpose }
	Connecting the phone to the project website is needed so that the user will be able to view their model in a VR environment. 

\paragraph[Input/Output sequence for this
feature]{\selectlanguage{english}\rmfamily\bfseries\color{black}
Input/Output sequence }
	The user scans the QR code on the website with a QR application on their phone. The model that is currently in the Forge viewer
	on the website will now be displayed on the phone. 

\paragraph[Design constraints of this
feature]{\selectlanguage{english}\rmfamily\bfseries\color{black} Design
constraints }
	The user must have a smartphone that is capable of using a QR scanning application. The phone also has access to an Internet 
	connection. 
	
	If the user opts to view the model in VR, then a VR solution is required for the user's device.

\paragraph[Performance requirements of this
feature]{\selectlanguage{english}\rmfamily\bfseries\color{black}
Performance requirements }
	The phone should be connected to the website within a few seconds. This might vary depending on how good of an Internet 
	connection the user currently has. 

\paragraph[Detailed functional requirements of this
feature]{\selectlanguage{english}\rmfamily\bfseries\color{black}
Detailed functional requirements }
	In order to provide VR functionality for an android device, the software must first establish a connection. Using a QR scanner, 
	the android device will be linked to a mobile version of the viewer displayed in stereoscopic format. This requires a stable 
	Internet connection in order to deliver the content to the device.

%end feature 3

\subsubsection[{Data Management}]{\selectlanguage{english}\rmfamily\bfseries\color{black}  
	User access to files on A360 and Fusion 360
}

\paragraph[Introduction/Purpose of this
feature]{\selectlanguage{english}\rmfamily\bfseries\color{black}
Introduction/Purpose }
	Users that have files in A360 or in Fusion 360 should be able to gain access through those files from the
	website.

\paragraph[Input/Output sequence]{\selectlanguage{english}\rmfamily\bfseries\color{black}
Input/Output sequence }
	The user should be able to somehow identify that they want to gain access to their files on either A360
	or Fusion 360. Using the Forge Data Management API the user should then have access to both of these
	Cloud storage sites.

\paragraph[Design constraints]{\selectlanguage{english}\rmfamily\bfseries\color{black} Design
constraints }
	Must have an account with either A360 or Fusion 360. 
	
	Must conform to API general use cases

\paragraph[Performance requirements]{\selectlanguage{english}\rmfamily\bfseries\color{black}
Performance requirements }
	The files should load in a viewable format.
	
	The files should only be visible if they are viewable.

\paragraph[Detailed functional requirements]{\selectlanguage{english}\rmfamily\bfseries\color{black}
Detailed functional requirements }
	The website give users the option to access their A360 and Fusion 360 accounts. This should be done through
	the use of the Forge Data Management API. Once the user has access to these accounts they should then be 
	able to obtain the files they want to use and upload them to the Forge viewer.

%End of feature 4

\subsubsection[{Model Derivative}]{\selectlanguage{english}\rmfamily\bfseries\color{black} \rmfamily\bfseries\color{black}  
	User can convert their files.
}


\paragraph[Introduction/Purpose of this
feature]{\selectlanguage{english}\rmfamily\bfseries\color{black}
Introduction/Purpose }
	With the use of the Forge Model Derivative API users should be able to convert any source files that they have into SVF format
	so that they can be displayed in the Forge viewer.

\paragraph[Input/Output sequence]{\selectlanguage{english}\rmfamily\bfseries\color{black}
Input/Output sequence }
	The user will upload a source file that they want to convert. The model derivative API will take the source file and return the 
	converted file.

\paragraph[Design constraints]{\selectlanguage{english}\rmfamily\bfseries\color{black} Design
constraints }
	Must use appropriate CAD source files. 

\paragraph[Performance requirements]{\selectlanguage{english}\rmfamily\bfseries\color{black}
Performance requirements }
	The program should clearly display viewable files.

\paragraph[Detailed functional requirements]{\selectlanguage{english}\rmfamily\bfseries\color{black}
Detailed functional requirements }
	The website must allow users to take their CAD source files and convert them to a file that will be usable in the
	Forge viewer.    

%End of feature 5

\subsubsection[{Hardware Detection}]{\selectlanguage{english}\rmfamily\bfseries\color{black}  
	Hardware Detection
}

\paragraph[Introduction/Purpose of this
feature]{\selectlanguage{english}\rmfamily\bfseries\color{black}
Introduction/Purpose }
	This feature serves as a way for the software to understand the hardware its being run on. This will be the foundation for 
	giving a user feedback on potential experience viewing.

\paragraph[Input/Output sequence]{\selectlanguage{english}\rmfamily\bfseries\color{black}
Input/Output sequence }
	The user connects their phone to Vrok-It through the use of a QR scanner. After connection it should verify what type of 
	device that the user has connected.

\paragraph[Design constraints]{\selectlanguage{english}\rmfamily\bfseries\color{black} Design
constraints }
	Reliance on the user giving permission to allow the software to get information about hardware specifications. Similar to 
	Androids permissions the user many not want to give information out so our hardware detection may be hindered and 
	unable to function at all. 

\paragraph[Performance requirements]{\selectlanguage{english}\rmfamily\bfseries\color{black}
Performance requirements }
	Needs to be able to obtain hardware specification as fast as possible to give users feedback immediately. The sooner 
	our software understands the user's hardware the faster it can give a recommendation about optimal viewing. 

\paragraph[Detailed functional requirements]{\selectlanguage{english}\rmfamily\bfseries\color{black}
Detailed functional requirements }
	The software must be able to detect the user's smartphone when it connects using the QR code provided. When a 
	connection is made, the software will then detect the device, verify it is supported. If device is known, the software 
	then will use presets for the viewer on the device, otherwise it should alert the user about incompatibility and 
	performance conflicts. These alerts will inform the user that the device will not operate optimally with the project. 

%end of feature 6

\subsubsection[{Google Cardboard}]{\selectlanguage{english}\rmfamily\bfseries\color{black} 
	A viewable model in VR 
}

\paragraph[Introduction/Purpose of this
feature]{\selectlanguage{english}\rmfamily\bfseries\color{black}
Introduction/Purpose }
	View the model on the user's device in VR using peripherals such as Google Cardboard.   

\paragraph[Input/Output sequence]{\selectlanguage{english}\rmfamily\bfseries\color{black}
Input/Output sequence }
	Input: A functioning model uploaded to the project site.
	
	Output: A model displayed on the user's device.

\paragraph[Design constraints]{\selectlanguage{english}\rmfamily\bfseries\color{black} Design
constraints }
	Depends heavily on hardware on the device that the user is using. Devices with low-end hardware will likely not be able to 
	display models that are large or have a lot of detail very well.  

\paragraph[Performance requirements]{\selectlanguage{english}\rmfamily\bfseries\color{black}
Performance requirements }
	The models need to be able to be manipulated by the viewer. If they are too large, the models will cause errors if they are 
	modified within the viewer.  

\paragraph[Detailed functional requirements]{\selectlanguage{english}\rmfamily\bfseries\color{black}
Detailed functional requirements }
	The VR model viewer displayed on the Android device must be supported on Google Cardboard's lenses. The model must 
	be tracked properly in the model viewer. The model viewer on the mobile device can be controlled by the webpage.  

%end of feature 7

\subsubsection[{Website Redesign}]{\selectlanguage{english}\rmfamily\bfseries\color{black} }

\paragraph[Introduction/Purpose of this
feature]{\selectlanguage{english}\rmfamily\bfseries\color{black}
Introduction/Purpose }
	In order to accommodate for new features, we would like to redesign the landing, and key interaction areas of the website. 
	This will allow for more creative development as the project progresses, as well as improve the look of the project as a whole.


\paragraph[Input/Output sequence]{\selectlanguage{english}\rmfamily\bfseries\color{black}
Input/Output sequence }
	Input: Website source files and control. 
	
	Output: An updated and structured web layout.

\paragraph[Design constraints]{\selectlanguage{english}\rmfamily\bfseries\color{black} Design
constraints }
	This website must support the Forge API, the Forge viewer, and must be viewable on mobile. 

\paragraph[Performance requirements]{\selectlanguage{english}\rmfamily\bfseries\color{black}
Performance requirements }
	The updated website must have support for the same sources as the Vrok-it website.

\paragraph[Detailed functional requirements]{\selectlanguage{english}\rmfamily\bfseries\color{black}
Detailed functional requirements }
	The website must be able to properly display the project. The updated site will include a new landing,
	separate pages for the project, as well as being able to content as we expand. Key functionality such as uploading
	a model file and viewing the file should be accessible as soon as the user loads the site.   

%end of feature 8

\clearpage
\section[GANTT CHART/SIGNATURES]{\selectlanguage{english}\rmfamily\bfseries\color{black}
GANTT CHART \& SIGNATURES}


	

%\begin{comment}
\newpage
\null
\vfill
%\clearpage\setcounter{page}{1}\pagestyle{Convertvi}
\begin{flushleft}


	\begin{Form}
		%\digsigfield{14cm}{3cm}{Sign Here} 	
		\rule{5in}{.4mm}\\
			Patti Vrobel\hspace{60ex}Date
	\end{Form}
		
	\vspace{1cm}	
	
	\rule{5in}{.4mm}\\
	Griffin Gonsalves\hspace{55ex}Date
		
	\vspace{1cm}	
	
	\rule{5in}{.4mm}\\
	Shawn Cross\hspace{59ex}Date
	
	\vspace{1cm}	
	
	\rule{5in}{.4mm}\\
	Paul Kwak\hspace{61ex}Date

\end{flushleft}
%\end{comment}

\end{document}
